% ================================================================================
%
%		ANSCHREIBEN (XeLaTeX)
%		=====================
%
% ================================================================================
\documentclass[10pt,a4paper]{scrartcl}

\usepackage[ngerman]{babel}					% n�tig f�r deutsche Silbentrennung
\usepackage[utf8]{inputenc}					% n�tig f�r deutsche Umlaute
\usepackage{fontspec} 						% n�tig zum laden der Fonts
\usepackage[big]{layaureo} 					% n�tig f�r bessere DIN A4 Seitenformattierung
\usepackage{scrpage2}						% n�tig f�r die Kopfzeile
\usepackage{ifthen}							% n�tig f�r �berpr�fung Ansprechpartner bei \anschrift


% ================================================================================
%
% 		Konfiguration:
%		==========
%
% ================================================================================
\defaultfontfeatures{Mapping=tex-text}			% Schriftarten
\setmainfont[
	SmallCapsFont = Fontin-SmallCaps.otf,
	BoldFont = Fontin-Bold.otf,
	%ItalicFont = Fontin-Italic.otf
]{Fontin.otf}
\font\fb=''[cmr10]''

\pagestyle{scrheadings}						% Kopfzeile
\setlength{\headheight}{3cm}
\renewcommand*{\headfont}{\normalfont}


% ================================================================================
%
%		Eigene Kommandos:
%		================
%
% ================================================================================
\newcommand{\anschriftA}[5]{					% f�gt die eigene Anschrift ein
	\begin{flushright}						%
		\textsc{#1} \\						%	1) Eigener Name
		#2 \\ #3 \\							%	2) Strasse + Hausnummer
		#4 \\ #5							%	3) Postleitzahl + Stadt
	\end{flushright}							%	4) Telefonnummer
}										%	5) e-Mail-Adresse

\newcommand{\anschriftB}[4]{					% f�gt Betriebs-Anschrift ein (optionaler Ansprechpatner)
	\begin{flushleft}							%
		\textsc{#1} \\						%	1) Name des Betriebs
		\ifthenelse{\equal{\detokenize{#2}}{}}		%	2) Ansprechpartner (optional)
			{} {z.Hd. #2 \\}					%	3) Strasse + Hausnummer
		#3 \\ #4							%	4) Postleitzahl + Stadt
	\end{flushleft}							%
	\vspace{0,5cm}							%
}										%

\newcommand{\datum}{						% f�gt das aktuelle Datum rechtsb�ndig ein
	\begin{flushright}						%
		\today							%
	\end{flushright} \vspace{0,5cm}				%
}										%

\newcommand{\reftitle}[1]{						% Referenz zur Stelle, wird extra dick geschrieben
	\begin{flushleft}							%
		\textbf{#1}							%	1) Zu schreibender Titel
	\end{flushleft} \vspace{0,5cm}				%
}										%

\newcount\tmp
\newcommand{\freeline}[1]{					% f�gt soviele Leerzeilen ein wie angegeben
	\tmp=0 \loop							%
		\advance\tmp by 1					%	1) Anzahl einzuf�gender Leerzeilen
		\leavevmode\newline					%
	\ifnum\tmp<#1 \repeat \\					%
}										%


% ================================================================================
%
%		Kopfzeile - Pers�hnliche Anschrift:
%		==========================
%
% ================================================================================
\ohead{
\anschriftA{Tobias Hahnen}
		{Greta-Rothe-Stra\ss{}e 32}
		{47443 Moers}
		{+49 175 4415445}
		%{+49 2841 502053}
		{tobias.hahnen@stud.hn.de}
}


\begin{document}


% ================================================================================
%
%		Firmenanschrift:
%		============
%
% ================================================================================
\anschriftB{thyssenkrupp Presta M\"ulheim GmbH}
		{Frau Caroline Navrade}
		{Sommerfeld 22-28}
		{45481 M\"ulheim an der Ruhr}


% ================================================================================
%
%		Aktuelles Datum:
%		=============
%
% ================================================================================
\datum


% ================================================================================
%
%		Referenztitel:
%		==========
%
% ================================================================================
\reftitle{Bewerbung als Praktikant im Bereich Web-Entwicklung \\ Stellen-ID DE\_800902\_CT-PSTMH00025}


% ================================================================================
%
% 		Anschreibungstext:
%		==============
%
% ================================================================================
\begin{flushleft}
Sehr geehrte Frau Navrade, \freeline{1}

mein Name ist Tobias Hahnen und ich bin derzeit eingeschriebener Student der Informatik im 5. Semester an der Hochschule Niederrhein.
Durch mein Studium habe ich bisher viele Einblicke in die unterschiedlichen Bereiche der Informatik erhalten und mir ein breites Spektrum an Wissen aneignen k\"onnen.
Die erlernten F\"ahigkeiten, der f\"ur mich besonders interessanten Teilgebiete, habe ich dar\"uber hinaus als angestellte Studentische Hilfskraft und in meiner Freizeit weiter vertieft. \freeline{1}

Als Studentische Hilfskraft an der Universit\"at Duisburg-Essen konnte ich, neben meinem vorherigen, schulischen Basiswissen in HTML und JavaScript, einen gr\"o\ss{}eren Teil der professionellen Web-Entwicklung in einem anwendungsspezifischem Projekt erlernen. Dazu z\"ahlen der Einsatz von Web-Servern wie Nginx unter Verwendung von Skriptsprachen wie PHP und dar\"uber hinaus erste Erfahrungen mit CSS und der damit verbundenen Gestaltung von Webseiten. \\
In F\"achern wie Web-Entwicklung und Interaktive Systeme an der Hochschule Niederrhein konnte ich auf diesem Wissen aufbauen und habe, neben dem Erlernen von neuen Techniken und theoretischen Grundlagen, in den praxisbezogenen Aufgaben viel Ehrgeiz in die Entwicklung der geforderten Systeme gesteckt.
Besonders bin ich dabei an Techniken wie REST, aber auch asynchroner Programmierung interessiert, wobei ich meine F\"ahigkeiten dahingehend gerne weiter ausbauen m\"ochte. \freeline{1}

Daher bin ich an einem Praktikum in diesem Bereich sehr interessiert, auch da mein Interesse und meine F\"ahigketen \"uber das in der Universit\"at und Hochschule vermittelte Wissen hinausragt.
Aber auch im Rahmen des im kommenden 6. Semester geplanten Pflichtpraktikums m\"ochte ich mich gerne bei Ihnen f\"ur ein Vollzeit-Praktikum ab dem 01.03.2020 bewerben.
Dadurch, dass es sich bei Ihnen um ein gr0\ss{}es, etabliertes Unternehmen handelt, denke ich, dass ich sehr viel erlernen kann, Ihnen aber gleichzeitig meine bisherigen F\"ahigkeiten unter Beweis stellen kann um die Entwicklung der jeweiligen Anwendungen optimal durchzuf\"uhren. \freeline{1}

Im Zuge dessen, freue ich mich von Ihnen zu h\"oren. In einem pers\"ohnlichen Gespr\"ach w\"urde ich Sie gerne von meiner Motivation und meinen genannten, sowie weiteren, F\"ahigkeiten \"uberzeugen. Wenn Sie bis dahin noch weitere Fragen haben sollten, stehe ich Ihnen gerne telefonisch als auch per e-Mail zur Verf\"ugung. \freeline{1}

Mit freundlichen Gr\"u\ss{}en \freeline{2}

Tobias Hahnen \freeline{4}

\textsc{Anlagen}: Lebenslauf, Abiturzeugnis
\end{flushleft}


\end{document}
