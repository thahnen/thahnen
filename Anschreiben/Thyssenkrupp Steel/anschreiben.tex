% ================================================================================
%
%		ANSCHREIBEN (XeLaTeX)
%		=====================
%
% ================================================================================
\documentclass[10pt,a4paper]{scrartcl}

\usepackage[ngerman]{babel}					% n�tig f�r deutsche Silbentrennung
\usepackage[utf8]{inputenc}					% n�tig f�r deutsche Umlaute
\usepackage{fontspec} 						% n�tig zum laden der Fonts
\usepackage[big]{layaureo} 					% n�tig f�r bessere DIN A4 Seitenformattierung
\usepackage{scrpage2}						% n�tig f�r die Kopfzeile
\usepackage{ifthen}							% n�tig f�r �berpr�fung Ansprechpartner bei \anschrift


% ================================================================================
%
% 		Konfiguration:
%		==========
%
% ================================================================================
\defaultfontfeatures{Mapping=tex-text}			% Schriftarten
\setmainfont[
	SmallCapsFont = Fontin-SmallCaps.otf,
	BoldFont = Fontin-Bold.otf,
	%ItalicFont = Fontin-Italic.otf
]{Fontin.otf}
\font\fb=''[cmr10]''

\pagestyle{scrheadings}						% Kopfzeile
\setlength{\headheight}{3cm}
\renewcommand*{\headfont}{\normalfont}


% ================================================================================
%
%		Eigene Kommandos:
%		================
%
% ================================================================================
\newcommand{\anschriftA}[5]{					% f�gt die eigene Anschrift ein
	\begin{flushright}						%
		\textsc{#1} \\						%	1) Eigener Name
		#2 \\ #3 \\							%	2) Strasse + Hausnummer
		#4 \\ #5							%	3) Postleitzahl + Stadt
	\end{flushright}							%	4) Telefonnummer
}										%	5) e-Mail-Adresse

\newcommand{\anschriftB}[4]{					% f�gt Betriebs-Anschrift ein (optionaler Ansprechpatner)
	\begin{flushleft}							%
		\textsc{#1} \\						%	1) Name des Betriebs
		\ifthenelse{\equal{\detokenize{#2}}{}}		%	2) Ansprechpartner (optional)
			{} {z.Hd. #2 \\}					%	3) Strasse + Hausnummer
		#3 \\ #4							%	4) Postleitzahl + Stadt
	\end{flushleft}							%
	\vspace{0,5cm}							%
}										%

\newcommand{\datum}{						% f�gt das aktuelle Datum rechtsb�ndig ein
	\begin{flushright}						%
		\today							%
	\end{flushright} \vspace{0,5cm}				%
}										%

\newcommand{\reftitle}[1]{						% Referenz zur Stelle, wird extra dick geschrieben
	\begin{flushleft}							%
		\textbf{#1}							%	1) Zu schreibender Titel
	\end{flushleft} \vspace{0,5cm}				%
}										%

\newcount\tmp
\newcommand{\freeline}[1]{					% f�gt soviele Leerzeilen ein wie angegeben
	\tmp=0 \loop							%
		\advance\tmp by 1					%	1) Anzahl einzuf�gender Leerzeilen
		\leavevmode\newline					%
	\ifnum\tmp<#1 \repeat \\					%
}										%


% ================================================================================
%
%		Kopfzeile - Pers�hnliche Anschrift:
%		==========================
%
% ================================================================================
\ohead{
\anschriftA{Tobias Hahnen}
		{Greta-Rothe-Stra\ss{}e 32}
		{47443 Moers}
		%{+49 1578 9540383}
		{+49 02841 502053}
		{tobias.hahnen@stud.hn.de}
}


\begin{document}


% ================================================================================
%
%		Firmenanschrift:
%		============
%
% ================================================================================
\anschriftB{thyssenkrupp Steel Europe AG}
		{Frau Silvia Kapp}
		{Kaiser-Wilhelm-Stra\ss{}e 100}
		{47166 Duisburg}


% ================================================================================
%
%		Aktuelles Datum:
%		=============
%
% ================================================================================
\datum


% ================================================================================
%
%		Referenztitel:
%		==========
%
% ================================================================================
\reftitle{Initiativbewerbung f\"ur ein Hochschulpraktikum im Bereich IT/Anlagen- und Prozesstechnik}


% ================================================================================
%
% 		Anschreibungstext:
%		==============
%
% ================================================================================
\begin{flushleft}
Sehr geehrte Frau Kapp, \freeline{1}

mein Name ist Tobias Hahnen und ich bin derzeit eingeschriebener Student der Informatik im 5. Semester an der Hochschule Niederrhein.
Durch mein Studium habe ich bisher viele Einblicke in die unterschiedlichen Bereiche der Informatik erhalten und mir ein breites Spektrum an Wissen aneignen k\"onnen.
Die erlernten F\"ahigkeiten, der f\"ur mich besonders interessanten Teilgebiete, habe ich dar\"uber hinaus in meiner Freizeit und als angestellte Studentische Hilfskraft weiter vertieft. \freeline{1}

Besonders bin ich dabei an der hardwarenahen Programmierung mit C/C++ sowie der Speicher- und Rechenoptimierung von Softwareprozessen interessiert, wobei ich mein Wissen zu letzterem vor allem durch meine Anstellung als Studentische Hilfskraft des iPattern Institut f\"ur Mustererkennung der Hochschule Niederrhein weiter ausbauen und anwenden konnte. \\
Als Studentische Hilfskraft hatte ich die M\"oglichkeit, einen tiefen Einblick in die Bild- und Mustererkennung zu erlangen und meine F\"ahigkeiten in der Teamarbeit weiter zu festigen. \\
Au\ss{}erdem konnte ich, auch durch mein Engagement in meiner Freizeit, analytische F\"ahigkeiten im Bereich der Softwareentwicklung erwerben. \freeline{1}

Aufgrund dieser Qualifikationen, die ich allerdings auch gerne weiter ausbauen m\"ochte, und dem im Rahmen des kommenden 6. Semesters geplanten Pflichtpraktikums, m\"ochte ich mich initiativ f\"ur ein dreimonatiges Vollzeit-Praktikum ab dem 01.03.2020 bei Ihnen im Bereich der IT/Anlagen- und Prozesstechnik bewerben. \\
Ich sehe dadurch, dass Sie mir ein Praktikum bei ihnen gew\"ahren, die Chance, Ihnen meine F\"ahigkeiten zum Positiven unter Beweis stellen zu k\"onnen. Gleichzeitig h\"atte ich die M\"oglichkeit, mein vorhandenes Wissen sowohl anzuwenden als auch zu erweitern und dar\"uber hinaus neue und damit n\"utzliche Fertigkeiten zu erlernen. \freeline{1}

Im Zuge der Initiative meinerseits, freue ich mich von Ihnen zu h\"oren. In einem pers\"ohnlichen Gespr\"ach w\"urde ich Sie gerne von meiner Motivation und meinen genannten, sowie weiteren, F\"ahigkeiten \"uberzeugen. Wenn Sie bis dahin noch weitere Fragen haben sollten, stehe ich Ihnen gerne telefonisch als auch per e-Mail zur Verf\"ugung. \freeline{1}

Mit freundlichen Gr\"u\ss{}en \freeline{2}

Tobias Hahnen \freeline{4}

\textsc{Anlagen}: Lebenslauf, Abiturzeugnis
\end{flushleft}


\end{document}
