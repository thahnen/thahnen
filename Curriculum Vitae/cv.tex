% ================================================================================
%
%		Lebenslauf/ CV (XeLaTeX)
%
%		=> Zum kompilieren: $ xelatex cv.tex && biber cv.bcf && xelatex cv.tex && open cv.pdf
%		=> geht nicht in TeXShop wegen BibLaTeX und Biber!
%
% ================================================================================


\documentclass[10pt,a4paper]{article}

\usepackage[ngerman]{babel}				% nötig für deutsche Silbentrennung
\usepackage{marvosym}					%
\usepackage{fontspec} 					% nötig zum laden der Fonts
\usepackage{xunicode,xltxtra,url,parskip}		% nötig zur Formatierung
\RequirePackage{color,graphicx}			% nötig für Farben und Graphiken		!!!
\usepackage[usenames,dvipsnames]{xcolor}	%								!!!
\usepackage[big]{layaureo} 				% nötig für bessere DIN A4 Seiten-Formattierung
\usepackage{supertabular} 				%								!!!
\usepackage{titlesec}					% nötig für spezielle Abschnitte
\usepackage{hyperref}					% nötig für Hyperlinks (ua. innerhalb der PDF)
\usepackage[absolute]{textpos}
\usepackage[							% nötig für Einbindung von BibTeX-Dateien
	backend=biber,
	babel=hyphen,
	style=authortitle,
	minnames=5,
	maxnames=10
]{biblatex}


% Konfiguration BibTeX
\addbibresource{papers.bib}

% Konfiguration Hyperlinks
\definecolor{linkcolour}{rgb}{0,0.2,0.6}
\hypersetup{colorlinks,breaklinks,urlcolor=linkcolour, linkcolor=linkcolour}

% Konfiguration Schriften
\defaultfontfeatures{Mapping=tex-text}
\setmainfont[
	SmallCapsFont = Fontin-SmallCaps.otf,
	BoldFont = Fontin-Bold.otf,
	ItalicFont = Fontin-Italic.otf
]{Fontin.otf}
\font\fb=''[cmr10]''

% Konfiguration Abschnitte
\titleformat{\section}{\Large\scshape\raggedright}{}{0em}{}[\titlerule]
\titlespacing{\section}{0pt}{3pt}{3pt}

% Konfiguration DIN A4 Seite
\setlength{\TPHorizModule}{30mm}
\setlength{\TPVertModule}{\TPHorizModule}
\textblockorigin{2mm}{0.65\paperheight}
\setlength{\parindent}{0pt}
\pagestyle{empty}


\begin{document}


% ================================================================================
%
%		Titel (Vor- & Zuname)
%
% ================================================================================
\par{
	\centering{
		\Huge Tobias \textsc{Hahnen}
	}\bigskip\par
}


% ================================================================================
%
%		Wichtigste Informationen:
%		===================
%
%		- Persönliche Informationen
%		- Berufserfahrung
%		- Ausbildung
%
% ================================================================================
\section{Pers\"ohnliche Informationen}

\begin{tabular}{rl}
    \textsc{Geburtsdatum und -ort:}	& 11.07.1998 | Moers \\
    \textsc{Staatsangeh\"origkeit:}	& Deutsch \\
    \textsc{Adresse:}				& Westhoffstraße 4, 44791 Bochum \\
    \textsc{Mobilfunknummer:}		& +49 175 4415445 \\
    \textsc{e-Mail:}				& \href{mailto:tobias.hahnen@protonmail.ch}{tobias.hahnen@protonmail.ch}
\end{tabular} \\


\section{Berufserfahrung}

\begin{tabular}{r|p{11cm}}
	% Softwareentwickler / SonarLint Java Developer: SonarSource GmbH
	\textsc{April} 2023 		& Softwareentwickler, \textsc{SonarSource GmbH} \\
	- \textsc{Heute} 		& \emph{Product Development: SonarLint} \\
						& \footnotesize{\href{https://www.sonarsource.com/products/sonarlint/features/eclipse/}{SonarLint f\"ur Eclipse} und \href{https://www.sonarsource.com/products/sonarlint/features/jetbrains/}{SonarLint f\"ur IntelliJ}, statische Quelltextanalyse direkt in der Entwicklungsumgebung.} \\
	\multicolumn{2}{c}{} \\
	
	% Softwareentwickler / DevOps bzw. Developer Experience Engineer: VISUS Health IT GmbH
	\textsc{Januar} 2021 	& Softwareentwickler, \textsc{VISUS Health IT GmbH} \\
	- \textsc{M\"arz} 2023 	& \emph{R\&D: DevOps \& Developer Experience} \\
						& \footnotesize{Neugestaltung der Build-Infrastruktur f\"ur die JiveX-Produktpalette, Einf\"uhrung von automatisierten \href{https://www.qfs.de}{Benutzeroberfl\"achentests}, \"Uberarbeitung / Ausarbeitung des Entwicklungs- und Freigabe- und Patch-Prozess (inkl. Tooling).} \\
	\multicolumn{2}{c}{} \\
	
	% Praktikum: ThyssenKrupp Steel Europe AG
	\textsc{M\"arz} 2020 		& Praktikant: Softwareentwickler, \textsc{thyssenkrupp Steel Europe AG} \\
	- \textsc{Juni} 2020		& \emph{Anlagen- und Prozesstechnik} \\
						& \footnotesize{Softwareentwicklung (inkl. Analyse des bestehenden / Modellierung eines neuen System) zur automatischen Restmengenoptimierung einer SMS group Stranggie\ss anlage in der GWA Bruckhausen. Bestehend aus Schnittl\"angenoptimierung und Restendenminimierung.} \\
	\multicolumn{2}{c}{} \\

	% SHK: HS Niederrhein
 	\textsc{Februar} 2019	& Studentische Hilfskraft: Softwareentwickler, \textsc{Hochschule Niederrhein} \\
	- \textsc{Februar} 2020	& \emph{iPattern Institut f\"ur Mustererkennung} \\
						& \footnotesize{Labeling von Echtzeitdaten sowie Softwareentwicklung zum \href{https://github.com/thahnen/labelbox-export-minifier}{automatischen Testen} und \href{https://github.com/thahnen/labelbox-scripts-etc}{Auswerten} dieser. Softwareentwicklung f\"ur ein \href{https://www.hs-niederrhein.de/ipattern/nachrichten-detailseite/?tx_news_pi1\%5Bnews\%5D=9545\&cHash=3202e26ca1ce23d3b3231df1f5c5a573}{Projekt} im Rahmen der medizinischen Bild- / Mustererkennung. Recherche sowie \href{https://github.com/thahnen/elastix-scripts-etc}{Softwareentwicklung} f\"ur ein Projekt im Rahmen der Medizintechnik.} \\
	\multicolumn{2}{c}{} \\
	
	% SHK: Uni Duisburg-Essen
	 \textsc{August} 2017	& Studentische Hilfskraft: Techniker, \textsc{Universit\"at Duisburg-Essen} \\
	 - \textsc{M\"arz} 2018	& \emph{Lehrstuhl Allgemeine Psychologie: Kognition} \\
						& \footnotesize{Umsetzung / Wartung eines Softwareprojekts zur Durchf\"uhrung eines \href{https://gdt.allgpsy.uni-due.de}{psychologischen Verfahrens}. Administration von Linux / MS Windows Servern. Administration von Mitarbeiter-Rechnern, Verwaltung mit Hilfe von MS Active Directory. Ausarbeitung der IT-Infrastruktur-Dokumentation.} \\
	 \multicolumn{2}{c}{} \\
\end{tabular} \\


\section{Ausbildung}

\begin{tabular}{r|p{11cm}}
%	BEHALTEN FÜR SPÄTER!
%	 \textsc{July} 2008 & Master of Science in \textsc{Finance}, \textbf{Bocconi University}, Milan\\
%	& 110/110 \small\emph{summa cum laude} | Major: Quantitative Finance\\
%	& Thesis: ``Sublinear and Locally Sublinear Prices'' | \small Advisor: Prof. Erio \textsc{Castagnoli}\\
%	& \normalsize \textsc{Gpa}: 28.61/30\hyperlink{grds}{\hfill | \footnotesize Detailed List of Exams} \\
%	& \\

	% Hochschule Niederrhein
	\textsc{M\"arz} 2018		& Informatik, \textbf{Hochschule Niederrhein}, Krefeld \\
	- \textsc{Dezember 2020}	& \emph{Abschluss: B.Sc. - Abschlussnote: 2.0} \\
						& Abschlussarbeit: ``Refactoring einer Identity Management Anwendung des Gemeinschaftslabors Informatik'' \\
	\multicolumn{2}{c}{} \\
	
	% Universität Duisburg-Essen
	\textsc{Oktober} 2016	& Angewandte Informatik, \textbf{Universit\"at Duisburg-Essen}, Duisburg \\
	- \textsc{M\"arz} 2018	& \emph{Schwerpunkt: Ingeneursinformatik} \\
	\multicolumn{2}{c}{} \\
	
	% Grafschafter Gymnasium Moers
	\textsc{Herbst} 2008		& Abitur, \textbf{Grafschafter Gymnasium}, Moers \\
	- \textsc{Juli} 2016		& \emph{Abschlussnote: 2.0} \\
\end{tabular} \\


% ================================================================================
%
%		Weitergehende Informationen und Skills:
%		==============================
%
%		- Publikationen (aus BibTeX-Datei)
%		- Sprachen
%		- IT-Kompetenzen
%		- Weitere Kompetenzen
%		- Ehrenamtliche Tätigkeiten
%		- Interessen und Aktivitäten
%
% ================================================================================
\nocite{*}
\printbibliography[title={Publikationen}]


\section{Sprachen}

\begin{tabular}{rl}
	\textsc{Deutsch:}		& Muttersprache \\
	\textsc{Englisch:}		& Fliessend (C1) \\
\end{tabular} \\


\section{IT-Kompetenzen}

\begin{tabular}{rl}
	Programmiersprachen:	& Python, Kotlin / Groovy / Java, C / C++, SQL, PHP, JavaScript, \\
						& Visual Basic Classic \\
	Build-Automatisierung:	& Gradle (inkl. Plugin-Entwicklung), Maven, Ant, Jenkins, \\
						& GitHub Actions, CMake \\
	Applikationen:			& SonarQube, GitHub, BitBucket / Jira / Confluence, IntelliJ / \\
						& Eclipse (inkl. Plugin-Entwicklung) \\
	Programmbibliotheken:	& CherryPy, jUnit, JavaFX, SQLite \\
	Technologien:			& Git, Mercurial, Legacy Code, REST,  Docker \\
	Betriebssysteme:		& UNIX-basiert (macOS, Linux, QNX), MS Windows \\
	Weitere F\"ahigkeiten:	& {\fb \LaTeX}\setmainfont[SmallCapsFont=Fontin-SmallCaps.otf]{Fontin.otf}, Markdown, Scrum / Kanban \\
\end{tabular} \\


\section{Weitere Kompetenzen}

\begin{tabular}{rl}
	Prozesse:				& Entwicklungs- / Freigabe- / Rollout- / Patch-Prozesse \\
	Versionsverwaltung:		& Repository-Strategien, Modularisierungs-Konzepte, Monolithen \\
						& und Microservices \\
	Tooling:				& Technische Dokumentation, Versionsverwaltung \\
\end{tabular} \\


\section{Ehrenamtliche Aktivit\"aten}

\begin{tabular}{rl}
	% Fachschaftsrat Informatik
	\textsc{Winter} 2016		& Fachschaftsrat Informatik \\
	- \textsc{M\"arz} 2018	& \emph{Universit\"at Duisburg-Essen} \\
	& \\
	
	% Code for Niederrhein Lead
	\textsc{Winter} 2014		& Leiter des \textbf{Code for Niederrhein} Lab: \\
	- \textsc{Sommer} 2018	& \emph{Open Knowledge Foundation} \\
\end{tabular} \\


\section{Interessen}
Buildsysteme, Prozessoptimierung, Open-Source-Software, Historische Computersysteme, \\
Raumfahrttechnik / -Geschichte, Fotografie, Pen \& Paper, Gesellschaftsspiele


\end{document}
